\documentclass[11pt,a4paper,notitlepage,onecolumn,twoside]{article}

\usepackage{amsmath}
\usepackage{amssymb}
\usepackage{amsthm}

\newtheorem{definition}{Definition}
\newtheorem{example}{Example}
\newtheorem{formula}{Formula}
\newtheorem{problem}{Problem}

\title{Identity on the Web \\ Notes}
\author{Wouter Beek \and Stefan Schlobach}

\begin{document}

\maketitle

\section{Introduction}

Identity is a problem on the Web. It is a very strong notion when used in an
open world. Using non-identity relations to assert similarity relations
hampers reasoning and often does not explain the aspects with respect to
which similarity is asserted.\footnote{Concepts can be similar in several
ways and dissimilar in other ways.}

\section{Identity}

According to the standard definition, $a$ and $b$ are identical
iff they co-designate [REF]. Co-designation is often cyclicly defined
in terms of indiscernability. The principle of the indiscernability of
identicals \ref{eq:leibniz_law} (also known as Leibniz's Law) 

\begin{formula}[Leibniz' Law]
\label{eq:leibniz_law}
$\forall a,b: a = b \rightarrow \forall \phi: \phi(a) \leftrightarrow \phi(b)$
\end{formula}

The converse of the principle of the indiscernability of identicals,
the principle of the identity of indiscernables
\ref{eq:principle_of_the_identity_of_indiscernables}, is vacuously true.

\begin{formula}[Principle of the identity of indiscernables]
\label{eq:principle_of_the_identity_of_indiscernables}
$\forall a,b: \forall \phi: \phi(a) \leftrightarrow \phi(b) \rightarrow a = b$
\end{formula}

The substitutivity of identicals \ref{eq:substitutivity_principle}
does not hold for intensionalilies (example \ref{eq:hespherus_phosphorus})
or modal contexts (example \ref{eq:substitution_modality}).

\begin{formula}[Substitutivity principle]
\label{eq:substitutivity_principle}
$a = b \rightarrow \phi \rightarrow [a \\ b]\phi$
\end{formula}

\begin{example}[Substitution of intensions]
\begin{subequations}
\label{eq:hespherus_phosphorus}
\begin{align}
\text{Hesperus contains eight letters.}\\
\text{Phosphorus contains eight letters.\footnote{Hespherus is the Evening Star; Phosphorus is the Morning Star.}}
\end{align}
\end{subequations}
\end{example}

\begin{example}[Substitution under modality]
\begin{subequations}
\label{eq:substitution_modality}
\begin{align}
\text{It is a necessary truth that 9 is larger than 7.}\\
\text{It is a necessary truth that 9 is the number of planets.\footnote{Pluto, anyone?}}
\end{align}
\end{subequations}
\end{example}

The relevance of the substitutivity principle to the identity relation
is unclear to me, since it only seems to restrict the number of deductions
involving identity, but not the validity of the identity relation itself.

\section{Identity on the Web}

Existing literature:
(1) A replacement of identity with similarity.
(2) A 

\begin{problem}
Some properties express propositions (e.g., $birth-year(wouter-beek, 1983)$),
but others express aspects of formulations of propositions
(e.g., $birth-year(wouter-beek, 1983)$ was asserted in 2013)
or labels/literal that are not identical but that do not necessarily
conflict (e.g. two descriptions formulated in different natural languages).

The indiscernability that comes with an identity relation is restricted to
properties that express propositions.
\end{problem}

\begin{problem}
In practice we come accross notational differences,
e.g. $Length(c_1, "8cm"^^xsd:string)$ and
$\exists x: Length(c_2,x) \land Value(x, 80^^xsd:int) \land Scale(x, "mm"^^xsd:string)$
\end{problem}

\section{Approach I}

We define the shared properties of two individuals.

\begin{definition}
\begin{equation}
\mathbb{P}(x,y) = \{ P \vert \exists P (xPz \land yPz) \}
\end{equation}
\end{definition}

Then we define the upper and lower boundary for the \emph{sameAs} relation.

\begin{definition}
\begin{align}
\overline{sameAs}(x,y) \  \iff \  sameAs(x,y) \land
    \forall u,v (\mathbb{P}(u,v) \subseteq \mathbb{P}(x,y) \rightarrow sameAs(u,v)) \\
\underline{sameAs}(x,y) \  \iff \  sameAs(x,y) \land
    \forall u,v (\mathbb{P}(u,v) \superseteq \mathbb{P}(x,y) \rightarrow sameAs(u,v))
\end{align}
\end{definition}

\section{Approach II}

How to handle:
\begin{itemize}
\item \textbf{Properties of arbitrary depth.} Properties can have arbitrary
      depth. For instance I am living in a city that is located in a country
      that is part of the European Union. Deep properties traverse
      (non-linked) resources or blank nodes.
\item \textbf{Property hierarchy} Should subproperties be treated
      as separate properties, not at all, or in a special way?
\item \textbf{Graphyness} An RDF graph is assumed to be a `real' graph,
      i.e., RDF properties cannot be RDF nodes.
\end{itemize}

We are given an equivalence relation $\approx$ which is induced
from the given set of triples. The properties that are used to construct
the quivalence relation can be freely chosen, although obvious choices
involve \emph{owl:sameAs}, \emph{skos:somewhatKindaSimila}.

The universe of discourse $\mathbb{U}$ consists of the RDF nodes that appear
in the given triples.

We define a mapping from partition members onto $\mathcal{P}(\mathbb{P})$,
where $\mathbb{P}$ are the properties that occur in the given set of triples.
\footnote{Here we assume that no property is an RDF node.}

\begin{definition}[Predicates]
\begin{eqnarray}
f_{\mathbb{P}}([x]_{\approx}) = \{ P &\vert&
    \exists y_1 \in [x]_{\approx}, \exists z_1 (y_1Pz_1 \land \\
    & & \  \forall y_2 \in [x]_{\approx} \setminus \{ y_1 \}
    (\exists z_2 (y_2Pz_2 \rightarrow z_1 = z_2))) \} \nonumber
\end{eqnarray}
\end{definition}

Partition members are grouped based on a similarity metric.
We choose the Jaccard index for this, but the choise for the combination
of metrics and boundary values is arbitrary.

\begin{definition}[Groups]
\begin{equation}
[y]_{\approx} \in G_{[x]_{\approx}} \  \text{iff} \ 
\frac{\vert f_{\mathbb{P}}([x]_{\approx}) \cap f_{\mathbb{P}}([y]_{\approx}) \vert}
     {\vert f_{\mathbb{P}}([x]_{\approx}) \cup f_{\mathbb{P}}([y]_{\approx}) \vert}
     > c
\end{equation}
\end{definition}

Based on the definition of groups we can identify the indiscernability
properties that characterize a collection of similar partition members.

\begin{definition}[Indiscernability]
\begin{equation}
\text{The members of} \  G_{[x]_{\approx}} \  \text{are} \ 
\underset{\substack{x' \in [x]_{\approx}}}{\operatorname{\bigcap}} f_{\mathbb{P}}([x']_{\approx})
\text{-indiscernable}.
\end{equation}
\end{definition}

We define the indiscernability set for a partition member as follows:

\begin{definition}[Indiscernability]
\begin{equation}
f_{\mathbb{I}}([x]_{\approx}) = \bigcap_{y \in G_{[x]_{\approx}}} f_{\mathbb{P}}(y)
\end{equation}
\end{definition}

\begin{definition}
\begin{equation}
G_{\approx} = \{ G_{[x]_{\approx}} \  \vert \  x \in \mathbb{U} \}
\end{equation}
\end{definition}

What is special about this approach is that we define (in)discernability on
the level of equivalence sets over individuals,
where normally this is defined on the level of individuals.

Observe that we are not talking about sets of indiscernable instances
(as in traditional approaches towards identity),
but about \emph{indiscernability descriptions}
(in terms of domain properties).
This means that $[x]_f([x]_{\approx}) = [x]_{\approx}$ is not true in the
general case. So for each group there can be instances in $[x]_{\approx}$
but not in $G_{[x]_{\approx}}$
(depending on the similarity strictures chosen).
But there can also be instances in $G_{[x]_{\approx}}$ that are not in
$[x]_{\approx}$.

\begin{definition}
\begin{equation}
\approx' = \{ \langle x, y \rangle \in \mathbb{U}^2 \  \vert \ 
    \forall P \in f_{\mathbb{I}}([x]_{\approx}), \exists z (xPz \leftrightarrow yPz) \}
\end{equation}
\end{definition}

\end{document}

