\subsection{Quality \& Approximation}
\label{sec:approximation}

Not all identity subrelations have the same quality.
Indeed, when we look at the subdivision into three `categories' above,
  we are able to distinguish between a lower approximation of identity,
  as the union of subrelations from the first category
  (definition \ref{def:identity_lower_approximation}),
  and a higher approximation of identity,
  as the union of subrelations from both the first and the second category
  (definition \ref{def:identity_higher_approximation}).

\begin{definition}[Lower approximation]
\label{def:identity_lower_approximation}
\begin{align}
  x \in \lowerapprox
\, \iff \,
    \setdef{y}{x \equiv_{\indp_{\approx}} y}
  \; \subseteq \;
    \approx\nonumber
\end{align}
\end{definition}

\begin{definition}[Higher approximation]
\label{def:identity_higher_approximation}
\begin{align}
  x \in \higherapprox
\, \iff \,
      \setdef{y}{x \equiv_{\indp_{\approx}} y}
    \, \cap \,
      \approx
  \; \neq \;
    \emptyset\nonumber
\end{align}
\end{definition}

\noindent Based on these approximations we can give
  the rough set representation $\pair{\lowerapprox}{\higherapprox}$
  of identity relation $\approx$ \cite{Pawlak1991}.\footnote{
    Relations are called `attributes' in rough set theory.
    They are functions that map to an arbitrary set of value labels.
    We only consider functions that map from binary input into
      the set of Boolean truth values, and therefore use the term
     `predicates' to denote these functions.
    We recognize that extensions to multi-valued logics
      would require a richer set of value labels.
  }
The quality of a rough set representation is given in
  definition \ref{def:quality}.
The intuition behind this quality measure is that the crispness
  of a set should be proportional to the quality
  of the identity relation on which it is based.
Since a consistently applied identity relation has relatively many
  partition sets that contain either
  no identity pairs (small value for $\lowerapprox$) or
  only identity pairs (large value for $\higherapprox$),
  a more consistent identity relation has a higher accuracy.

\begin{definition}[Quality]
\label{def:quality}
\begin{align}
  \alpha(\approx)
\, = \,
  \dfrac{
    \card{\underline{\sim}}
  }{
    \card{\overline{\sim}}
  }\nonumber
  %\card{\lowerapprox} / \card{\higherapprox}
\end{align}
\end{definition}

\noindent Now that we have a formal metric for quality,
  we can define the characteristics of an ideal identity relation.
Traditionally, the ideal identity relation ensures indiscernibility
  for all expressible properties in the language
  (principle \ref{principle:indiscernibility_of_identicals}).
According to this traditional view, an identity relation becomes of
  higher quality by considering more properties
  according to which two resources are verified to be indiscernible.
We give a different quality criterion by observing that
  for a given identity relation $\approx$,
  defined over a domain of resources $S_G$,
  we can define the notion of full discernibility
  (definition \ref{def:full_discernibility}).

\begin{definition}[Full discernibility]
\label{def:full_discernibility}
A domain $S_G$ is fully discernible with respect to
  a binary relation $\approx$ iff
\[
  \forall x,y \in S_G (
      x \in \equivset{y}
    \lor
      \indp_{\approx}(\equivset{x}) \neq \indp_{\approx}(\equivset{y})
    )
\]
\end{definition}

\noindent From definition \ref{def:full_discernibility}
  it is clear that a domain is fully discernable just in case
  there exists a binary relation $\approx$
  for which \mbox{$\alpha(\approx) = 1.0$}.

\begin{comment}
\begin{definition}[Higher \& lower approximation]
\label{def:higher_lower_approximation}
\begin{align}
  y \in [x]_H
\,\iff\,\\
  \exists u (
      \card{[u]_{\sim}}>1
    \,\land\,
      \mathbb{P}([u]_{\sim})=\mathbb{P}(\set{x,y})
  )\nonumber
\\
  y \in [x]_L
\,\iff\,
  \forall S \subseteq D (\\
      (\card{S}>1 \,\land\, \mathbb{P}(S) = \mathbb{P}(\set{x,y}))
    \,\rightarrow\,
      \exists s \in D (S=[s]_{\sim})
  )\nonumber
\end{align}
\end{definition}
\end{comment}
