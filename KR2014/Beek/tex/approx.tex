\subsection{Quality \& Approximation}
\label{sec:approximation}

Not all identity subrelations have the same quality.
Indeed, when we look at the subdivision into three `categories' above,
  we are able to distinguish between a lower approximation of identity,
  as the union of subrelations from the first category
  (definition \ref{def:identity_lower_approximation}),
  and a higher approximation of identity,
  as the union of subrelations from both the first and the second category
  (definition \ref{def:identity_higher_approximation}).

\begin{definition}[Lower approximation]
\label{def:identity_lower_approximation}
\begin{align}
  x \in \lowerapprox
\, \iff \,
    \setdef{y}{x \equiv_{\indp_{\approx}} y}
  \; \subseteq \;
    \approx\nonumber
\end{align}
\end{definition}

\begin{definition}[Higher approximation]
\label{def:identity_higher_approximation}
\begin{align}
  x \in \higherapprox
\, \iff \,
      \setdef{y}{x \equiv_{\indp_{\approx}} y}
    \, \cap \,
      \approx
  \; \neq \;
    \emptyset\nonumber
\end{align}
\end{definition}

\noindent Based on these approximations we can give
  the rough set representation $\pair{\lowerapprox}{\higherapprox}$
  of identity relation $\approx$.
The quality of a rough set representation is given in
  definition \ref{def:quality}.
The intuition behind this quality measure is that the crispness
  of a set should be proportional to the quality
  of the identity relation on which it is based.
Since a consistently applied identity relation has relatively many
  partition sets that contain either
  no identity pairs (small value for $\lowerapprox$) or
  only identity pairs (large value for $\higherapprox$),
  a more consistent identity relation has a higher accuracy.

\begin{definition}[Quality]
\label{def:quality}
\begin{align}
  \alpha(\approx)
\, = \,
  \dfrac{
    \card{\underline{\sim}}
  }{
    \card{\overline{\sim}}
  }\nonumber
  %\card{\lowerapprox} / \card{\higherapprox}
\end{align}
\end{definition}

