\section{Conclusion}
\label{sec:conclusion}

In this paper we presented a new approach for characterizing,
  extending, retracting, and assessing identity relations.
Our approach does this in purely qualitative terms, using schema semantics.

In section \ref{sec:introduction} we enumerated three research goals.
The first goal is met, since an indiscernibility partition characterizes
  identity subrelations based on the predicates $P$ (closed under identity)
  for which the pairs in that sets are indiscernible.
In this way we can distinguish between different types of identity
  by treating $P$ as a description of a (sub)set of identity pairs.
We suggest that the meaning of an identity relation and its subrelations
  is partially defined in its use,
  i.e., in the indiscernibility criteria it embodies.

The second goal is met, since the notion of a rough set allows us to
  distinguish between pairs that must be (lower approximation)
  and those that may be (higher approximation)
  % 'may' = 'not must not'
  in the identity relation.
If we want to add/remove pairs of the identity relation,
  we should not consider pairs of the former but only pairs of
  the latter kind.

The third goal is met, since the measure for rough set accuracy
  is based on the discernibility criteria of an identity set.
The crispness of the set is proportional to the quality of the
  identity relation, based on its semantic consistency.

