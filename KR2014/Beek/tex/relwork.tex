\section{Related work}
\label{sec:related_work}

Existing research suggests six different solutions for
  the problem of identity on the SW.

\textbf{[1] Introduce weaker versions of {\small \texttt{owl:sameAs}}}
  \cite{HalpinHayes2010,MccuskerMcguinness2010}.
Candidates for replacement are
  the SKOS concepts
  {\small \texttt{skos:related}} and {\small \texttt{skos:exactMatch}}.
The former is not transitive,
  thereby limiting the possibilities for reasoning.
The latter is transitive,
  but can only be used in certain contexts.
It is not defined in what contexts it can be used.

\textbf{[2] Restrict the applicability of identity relations}
  to specific contexts.
In terms of SW technology, identities are expected to hold
  within a named graph or within a namespace,
  but not necessarily outside of it \cite{HalpinHayes2010}.
\cite{Melo2013} has successfully used the Unique Names Assumption
  within namespaces in order to identify many (arguably) spurious
  identity statements.

\textbf{[3] Introduce additional vocabulary} that does not weaken but extend
  the existing identity relation.
\cite{HalpinHayes2010} mention an explicit distinction that can be made
  between mentioning a term and using a term.
Other possible extensions of {\small \texttt{owl:sameAs}} take
  the Fuzzyness and/or uncertainty of identity statements into account.

\textbf{[4] Use domain-specific identity relations}
  \cite{MccuskerMcguinness2010}.
Such domain-specific links are only locally valid,
  thereby limiting knowledge reuse.

\textbf{[5] Change the modeling practice}
  \cite{HalpinHayes2010,DingShinavierFininMcguinness2010}.
Introducing checks on editing operations violates
  one of the fundamental underpinnings of the SW:
  that anybody is allowed to say anything about anything
  \cite{AntoniouGrothHarmelenHoekstra2012}.

\cite{DingShinavierShangguanMcguinness2010} who show that
  network analysis of the occurrence of {\small \texttt{owl:sameAs}}
  in datasets can provide insights into the ways in which identity is used.
These latter endeavors have not yet been related to the semantics of
  identity.

