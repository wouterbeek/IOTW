\section{Introduction}
\label{sec:introduction}

Identity relations are a cornerstone of logic-based knowledge representation.
They allow to state and relate properties of an object
  using multiple names for that object, and conversely,
  they allow to infer that different names actually refer to the same object.

Identity relations are
  at the foundation of the Linked Open Data initiative
  and the Semantic Web (SW) in general \cite{BizerCyganiakHeath2007}.
The SW consists of sets of assertions that are published on the Web
  by different authors operating in different contexts,
  often using different names for the same object.
Identity relations allow the interlinking of these multiple descriptions
  of the same thing.

Identity is often understood as the sharing of all properties between
  two objects with different names (principle of indiscernibility),
  see Principle \ref{principle:indiscernibility_of_identicals}.
In the SW this traditional notion of identity is expressed by
  the \texttt{owl:sameAs} property.\cite{MotikPaterschneiderGrau2012}

\begin{principle}[Indiscernibility of identicals]
\label{principle:indiscernibility_of_identicals}
\begin{equation}
    a = b
  \rightarrow
    \forall \phi \in \Phi (\phi(a) = \phi(b)\nonumber)
\end{equation}
\end{principle}

\noindent According to the traditional semantics of the identity relation,
  identical terms can be replaced for one another in all non-modal contexts
  \emph{salva veritate}.
Practical uses of \texttt{owl:sameAs} are known to violate this
  strict condition
  \cite{HalpinHayes2010,HalpinHayesMccuskerMcguinnessThompson2010}.

On the SW,
  identity assertions are extra strong because of the Open World Assumption.
Stating that two objects are the same
  implies that from now on no new property can be stated about
  only one of those objects
Moreover, whether or not two objects share the absence of a property
  cannot be concluded based on the absence of a property assertion.

Improving on the existing semantics of identity,
  we have the following research goals:
\begin{enumerate}
\item In an identity relation the pairs all look the same.
      We want to characterize subrelations of an identity relation in terms
      of the predicates that are important in a particular context.
\item Based on an existing identity relation we want to give semantically
      motivated suggestions for extending or limiting the identity relation.
\item We want to assess the quality of an identity relation based on
      the consistency with which it is applied to the data.
\end{enumerate}



\subsection{Outline}

In the following section we survey existing work
  on the problem of identity on the SW.
In Section \ref{sec:approach} we present our approach;
  first at a very high level and then in more formal detail.
We illustrate the results of applying our formalism to
  a real-world dataset.
Section \ref{sec:conclusion} concludes.

