\section{Stating the problem}
\label{sec:stating_the_problem}

Identity is often understood as the sharing of all properties between
  two objects with different names. 
This statement is known as the principle of indiscernibility
  (see principle \ref{principle:indiscernibility_of_identicals})
  and has been attributed to Leibniz \cite{Forrest2010}.\footnote{
    Inverting the implication in
      principle \ref{principle:indiscernibility_of_identicals}
      results in the identity of indiscernibles,
      which is trivially true since identity with $\phi(b)$
      is one of the properties.
    }
% According to Forrest2010 this can be found in Gottfried Wilhelm Leibniz,
% Discourse on Metaphysics, section 9.
% There I find the following passage:
% \begin{quote}
% That every individual substance expresses the whole universe
%   in its own manner and that in its full concept is included
%   all its experiences together with all the attendant circumstances
%   and the whole sequence of exterior events.
% \end{quote}

\begin{principle}[Indiscernibility of identicals]
\label{principle:indiscernibility_of_identicals}
\begin{equation}
    a = b
  \rightarrow
    \forall \phi \in \Phi (\phi(a) = \phi(b)\nonumber)
\end{equation}
\end{principle}

\noindent Although the principle provides necessary and sufficient conditions
  for identity, it does not point toward an automated procedure
  for enumerating the extension of the identity relation.
Due to its circular nature, the set of properties includes
  ``being identical to $x$'' (for every object $x$).
Even though this principle does not
  allow a positive identification of identity pairs,
  it does provide an exclusion condition;
  namely objects that are known to not share some property
  are also known to not be identical.

\subsection{Generic problems of identity}

Identity poses several problems that are not specific to the SW.
Firstly, identity does not hold across (all) modal contexts,
  allowing Lois Lane to believe that Superman saved her,
  without knowing that Clark Kent saved her.

Secondly, identity seems to be a context-specific concept \cite{Geach1967},
  allowing two medicines to be the same chemical substance
  while not being the same commercial drug.

Thirdly, identity over time poses problems,
  since a ship may be considered the same
  even though all the components from which it is built
  have been changed over the course of time \cite{Lewis1986}.

Lastly, there is the problem of identity under counterfactual assertions
  such as ``If Obama would have been born outside the US,
  then he would not have been president of the US today.''\cite{Kripke1980}

\subsection{SW-specific problems of identity: Semantics}

Besides the generic problems of identity,
  there are problems specific to the SW.
The first SW-specific problem follows from its semantics;
  the second follows from its pragmatics.
The semantics for SW identity are given in definition \ref{def:owl_sameAs}.

\begin{definition}[Semantics of {\small \texttt{owl:sameAs}}]
\label{def:owl_sameAs}
\begin{equation}
    \langle a, b \rangle \in Ext(I({\small \texttt{owl:sameAs}}))
  \,\iff\,
    a = b\nonumber
\end{equation}
\end{definition}

\noindent In the context of the SW,
  identity assertions are extra strong because of the Open World Assumption.
Stating that two objects are the same
  implies that from now on no new property can be stated about
  only one of those objects
  (this follows from definition \ref{def:owl_sameAs} in combination with
  the principle of substitutivity \emph{salva veritate}).
  % Look this one up in
  % W.V.O. Quine, Quintessence, extensions, Reference and Modality, p. 378.
  % Looked this one up, interesting, but is not directly applicable here.
For instance, when one SW contributor claims that
  medicines $a$ and $b$ are the same
  based on them having the same chemical composition,
  this prohibits another SW contributor from stating that
  $a$ and $b$ are produced by different companies
  without her introducing an inconsistency.
Formulated in terms of
  principle \ref{principle:indiscernibility_of_identicals},
  on the SW the set $\Phi$ contains
  all properties that can possibly be expressed
  in the modeling language.
This set is obviously much bigger than the actual vocabulary
  of any in-use dataset.

Moreover, whether or not two objects share the absence of a property
  (i.e., a property of the form ``does not have the property $\phi$''),
  cannot be concluded based on the absence of a property assertion.
Such `negative knowledge' must be provided explicitly
  using class restrictions.

\subsection{SW-specific problems of identity: Pragmatics}

When we take the social component of the SW into account,
  we observe that modelers have different opinions about
  whether two objects are identical or not,
  because they operate in different contexts.
This is unlike many traditional uses of knowledge-bases,
  where a knowledge-base is rarely re-used outside its original
  context of construction.
As a consequence, SW modelers are known not to conform to
  the strict semantics of identity (definition \ref{def:owl_sameAs}),
  resulting in situations where one person
  claims two objects are identical
  whereas another considers them to be only (closely) related.
It is indeed unrealistic that all who contribute to the SW
  are able to quantify over the entire realm of possible properties
  (i.e., $\Phi$ in principle \ref{principle:indiscernibility_of_identicals}).

The SW community recognizes this tension between
  the semantics and pragmatics of identity.
At the Semantics for Big Data track at the AAAI Fall Symposium 2013
  \cite{SemanticsBigData2013}
  the problem of identity was considered one of the most
  pressing problems facing the Semantic Web today.

The pragmatics of the SW is encoded in the open-ended collection of
  common practices that effectuate the creation and usage of SW content.
An example of such an encoded common practice is the five `stars'
  of Linked Open Data (LOD) publishing \cite{Bernerslee2010}.
These five `stars' are effectively five \emph{maxims} that specify
  (part of) the pragmatics of LOD publishing,
  much like the Gricean maxims specify
  (part of) the pragmatics of natural language discourse \cite{Grice1989}.
The fifth `star' or maxim is given in \ref{eq:data_linking_maxim}.
\begin{principle}[Data linking maxim]
  \label{eq:data_linking_maxim}
  \begin{quote}
    Link your data to other people's data to provide context.
  \end{quote}
\end{principle}

\noindent Given that RDF links are often specified by using
  the {\small \texttt{owl:saveAs}} predicate term \cite{Void2011}
  % The VoID W3C Interest Group Note states that
  % ``RDF links often have the \texttt{owl:sameAs} predicate.''
  we conclude that the pragmatics of the SW
  aggravates the already problematic semantics of identity.

The pragmatics of the SW, observed in contemporary common practices,
  states that by asserting identity between objects,
  more context is added for those objects,
  since such identity links result in more facts being asserted
  about the same resource.
At the same time we see that the strict semantics of identity
  gets violated once the context in which an object was created
  is extended -- by identity assertions -- beyond its original context
  (see the section on the generic problems of identity).
This is true regardless of whether `context' is defined in terms of
  `intended use', `domain', `time', or `modality'.

Concluding, from the social point of view the requirements on SW modelers
  are unreasonably high when they are required to
  assert identity links in accordance with the semantics.
At the same time the pragmatics states that modelers should
  make those links in order to place their knowledge into context.



\subsection{Research goals}
\label{sec:research_goals}

Based on the above analysis, we can state the following desiderata for 
an identity relation that does not suffer from the problems stated above: 

\begin{enumerate}
\item In an identity relation the pairs all look the same.
      We want to characterize subrelations of an identity relation in terms
      of the predicates that are important in a particular context.
\item Based on an existing identity relation we want to give semantically
      motivated suggestions for extending or limiting the identity relation.
\item We want to assess the quality of an identity relation based on
      the consistency with which it is applied to the data.
\end{enumerate}
