\subsection{Approximation}
\label{sec:approximation}

In the previous section we partitioned a given identity relation $\sim$
  into subrelations that can be distinguished in terms of schema predicates
  (or $n$-depth paths of those predicates).
In this section we create an approximation of the identity relation.
This approximation will allow us to
  (1) give suggestions about which pairs to in/exclude from
      the identity relation, and
  (2) give an indicator for the quality of the identity relation.

For the approximation of the identity relation we use
  rough set theory \cite{pawlak_1991} to represent an approximation of
  a given identity relation $\sim$.
The domain for our rough set approach is the Cartesian product of $S_G$.
The set of predicates is the powerset of $P_G$.
%\footnote{
%  Relations are called `attributes' in rough set theory.
%  They are functions that map to an arbitrary set of value labels.
%  We only consider functions that map from binary input into
%    the set of Boolean truth values, and therefore use the term
%    `predicates' to denote these functions.
%}
%This means that we have a big number of primitives to work with
%  (quadratic in the number of constants;
%   exponential in the number of relations).

For an arbitrary binary relation $\sim$ we can define
  a higher (def. \ref{def:higher_approximation}) and
  a lower (def. \ref{def:lower_approximation}) approximation
  of that relation.
In definitions \ref{def:higher_approximation} and
  \ref{def:lower_approximation},
  $\mathbb{R}$ characterizes a similarity relation between resource pairs.
The intuition behind these definitions is that non-$\sim$-pairs
  that are similar to $\sim$-pairs should be in the higher approximation,
  whereas no $\sim$-pair that has a similar non-$\sim$-pair should be
  in the lower approximation.

\small
\begin{definition}[Higher \& lower approximation]
\begin{align}
\label{def:higher_approximation}
x \overline{\sim} y \, & \iff & \,
  \exists u,v (\pair{u}{v} \mathbb{R} \pair{x}{y} \,\land\, u \sim v)
\\
\label{def:lower_approximation}
x \underline{\sim} y \, & \iff & \,
  \forall u,v (\pair{u}{v} \mathbb{R} \pair{x}{y} \,\rightarrow\, u \sim v)
\end{align}
\end{definition}
\normalsize 

\begin{comment}
\small
\begin{definition}[Higher \& lower approximation]
\label{def:higher_lower_approximation}
\begin{align}
  y \in [x]_H
\,\iff\,\\
  \exists u (
      \cardinality{[u]_{\sim}}>1
    \,\land\,
      \mathbb{P}([u]_{\sim})=\mathbb{P}(\set{x,y})
  )\nonumber
\\
  y \in [x]_L
\,\iff\,
  \forall S \subseteq D (\\
      (\cardinality{S}>1 \,\land\, \mathbb{P}(S) = \mathbb{P}(\set{x,y}))
    \,\rightarrow\,
      \exists s \in D (S=[s]_{\sim})
  )\nonumber
\end{align}
\end{definition}
\normalsize
\end{comment}

\noindent Since we want to stay close to the traditional notion of identity,
  defined in terms of indiscernibility,
  we choose $\mathit{IND}(\mathcal{P}(P_G^n))$ as our similarity relation.

Figure 1 shows an example of the lower and higher
  approximations for a linkset.
Since in this figure a partition is only drawn when there is at least one
  identity pair that is indiscernible with respect to some set of
  predicates, the higher approximation amounts to the entire figure.
The lower approximation only consists of those partition sets that contain
  at least one identity pair, and that contain no non-identity pair.

\subsection{Quality}
\label{sec:quality}

Given the rough set representation $\pair{\underline{\sim}}{\overline{\sim}}$
  of identity relation $\sim$, we can calculate the accuracy of this
  approximation with equation \ref{eq:accuracy}.

\small
\begin{equation}
\label{eq:accuracy}
  \alpha(\sim)
\,=\,
  \cardinality{\underline{\sim}} / {\cardinality{\overline{\sim}}}
  %\dfrac{\cardinality{\underline{\sim}}}{\cardinality{\overline{\sim}}}
\end{equation}
\normalsize

The intuition behind the usefulness of equation \ref{eq:accuracy}
  is that the crispness of a set should be proportional to the quality
  of the identity relation on which it is based.
Since a consistently applied identity relation has relatively many
  partition sets that contain either
  no identity pairs (small value for $\overline{\sim}$) or
  only identity pairs (big value for $\underline{\sim}$),
  a more consistent identity relation has a higher accuracy.

Now that we have a formal metric for identity relation quality,
  we can define the characteristics of an ideal identity relation.
Traditionally the ideal identity relation ensures indiscernibility
  for all expressible properties in the language
  (the principle of the indiscernibility of identicals).
According to this traditional view an identity relation becomes of
  higher quality by considering more predicates (or ppms)
  according to which two resources are not allowed to be discernible.
We give a different quality criterion.

We observe that for a given equivalence relation $\sim$
  defined over a domain of resources $S_G$ we can define the notion of
  full discernibility:

\small
\begin{definition}[Discernible model]
\label{def:fully_discernible}
\begin{align}
& \text{A domain $S_G$ is fully discernible w.r.t. a binary relation $\sim$ iff}
\nonumber
\\
 & \forall x,y \in S_G (
    [x]_{\sim}=[y]_{\sim}
  \,\lor\,\,
    \mathbb{P}([x]_{\sim}) \neq \mathbb{P}([y]_{\sim})
  )
\end{align}
\end{definition}
\normalsize

\noindent From this definition it is clear that a domain of discourse
  is fully discernable just in case there exists a binary relation $\sim$
  such that \mbox{$\alpha(\sim) = 1.0$}.
