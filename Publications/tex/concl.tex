\section{Conclusion}
\label{sec:conclusion}

In this paper we have given an new approach for characterizing,
  extending/retracting, and assessing identity relations.
Our approach does this in purely qualitative terms, using schema semantics.
In contemporary ontology alignment and data linking activities nonsemantic
  aspects of resources play a role as well.
For instance similarity assessment for natural language labels is often
  used in data linking.

We think that the qualitative means of characterizing an identity relation
  are a useful addition to existing quantitative means.
Also, we think that it is more useful and viable to enrich existing
  identity relations in the LOD based on the semantics of the datasets
  in which they occur, than to introduce new relationships into SW languages.
Apart from the practical difficulties of teaching practitioners
  and transforming/enriching existing datasets, we suggest that the
  meaning of an identity (sub)relation is partially defined in its use,
  i.e., in the indiscernibility criteria it embodies.

For our approach it is not necessary to pose additional restrictions
  on a binary relation $\sim$.
The definitions in this paper apply to \verb|owl:sameAs| relations
  in the same way in which they apply to any other binary relation
  (e.g., \verb|skos:related|).

We are currently in the process of validating the above enumerated hypotheses.
The results of these evaluations are continuously being published on
  \verb|wouterbeek.com/identity-on-the-web|.
The website currently contains the automated results of all eighty IIMB
  alignments, drawn from the instance matching track of the
  OAEI 2012.
The website also refers to the publicly available Git repository
  \verb|github.com/wouterbeek/IOTW| where the implementation
  discussed in section \ref{sec:implementation} can be found.
