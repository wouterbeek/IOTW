\section{Conclusion}
\label{sec:conclusion}

In this paper we have given an new approach for characterizing,
  extending/retracting, and assessing identity relations.
Our approach does this in purely qualitative terms, using schema semantics.
In contemporary ontology alignment and data linking activities nonsemantic
  aspects of resources play a role as well.
For instance similarity assessment for natural language labels is often
  used in data linking.

In section \ref{sec:research_goals} we enumerated three research goals.
The first goal is met, since an indiscernibility partition characterizes
  identity subrelations based on the predicates $P$ (closed under identity)
  for which the pairs in that sets are indiscernible.
In this way we can distinguish between different types of identity
  by treating $P$ as a description of a (sub)set of identity pairs.

The second goal is met, since the notion of a rough set allows us to
  distinguish between pairs that must be (lower approximation)
  and those that may be (higher approximation)
  % 'may' = 'not must not'
  in the identity relation.
If we want to add/remove pairs of the identity relation,
  we should not consider pairs of the former but only pairs of
  the latter kind.

The third goal is met, since the measure for rough set accuracy
  is based on the discernibility criteria of an identity set.
The crispness of the set is proportional to the quality of the
  identity relation, based on its semantic consistency.

Our approach provides a new experimental design for evaluating
  hypotheses regarding identity relations that have not been
  evaluated before in terms of the semantics of the data.

We think that the qualitative means of characterizing an identity relation
  are a useful addition to existing quantitative means.
Also, we think that it is more useful and viable to enrich existing
  identity relations in the LOD based on the semantics of the datasets
  in which they occur, than to introduce new relationships into SW languages.
Apart from the practical difficulties of teaching practitioners
  and transforming/enriching existing datasets, we suggest that the
  meaning of an identity (sub)relation is partially defined in its use,
  i.e., in the indiscernibility criteria it embodies.

For our approach it is not necessary to pose additional restrictions
  on a binary relation $\approx$.
The definitions in this paper apply to {\small \texttt{owl:sameAs}} relations
  in the same way in which they apply to any other binary relation
  (e.g., {\small \texttt{skos:related}}).

We are currently in the process of validating the above enumerated hypotheses.
The results of these evaluations are continuously being published on
  \URL{wouterbeek.com/identity-on-the-web} and
\URL{github.com/wouterbeek/IOTW}.
