\section{Experimental design}
\label{sec:experimental_design}

For illustrative purposes we use the IIMB dataset that is used in the
  instance matching track of the 2012 Ontology Alignment Evaluation
  Initiative (OAEI) \cite{oaei_2012}.
This dataset consists of eighty ontologies $G_i$ (for $1 \leq i \leq 80$)
  that are linked to a single base ontology $G_0$.
A graph $G$ is the result of fully materializing the graph merge
  of $G_i$ (for some $1 \leq i \leq 80$) and $G_0$.
For each of these eighty linked ontologies a reference mapping is available.

In section \ref{sec:research_goals} we enumerated three research goals.
The first goal is met, since an indiscernibility partition characterizes
  subrelations based on the ppms $P$ for which the pairs in that sets
  are $cl_{\sim}(P)$-indiscernible.
In this way we can distinguish between different types of identity
  by treating $P$ as a description of a (sub)set of identity pairs.

The second goal is met, since the notion of a rough set allows us to
  distinguish between pairs that must be (lower approximation)
  and those that may be (i.e., ``not must not'', higher approximation)
  in the identity relation.
If we want to add/remove pairs of the identity relation,
  we should not consider pairs of the former but only pairs of
  the latter kind.

The third goal is met, since the measure for rough set accuracy
  is based on the discernibility criteria of an identity set.
The crispness of the set is proportional to the quality of the
  identity relation, based on its semantic consistency.

Our approach provides a new experimental design for evaluating
  hypothesis regarding identity relations that have not been
  evaluated before in terms of the semantics of the data.

\subsection{Hypotheses}
\label{sec:hypotheses}

Using this new approach the following hypothesis can be validated:

\begin{enumerate}
\item Take an \verb|owl:sameAs| relation and a \verb|skos:related|
        relation defined over the same domain.
      Merge them into a new binary relation $\sim$.
      Establishing the lower and higher approximation of $\sim$,
        the hypothesis is that pairs from \verb|owl:sameAs| occur more
        frequently in the lower boundary than pairs from \verb|skos:related|.
\item Take a set of alignment pairs, each of which is associated with
        a confidence measure between $0.0$ and $1.0$.
      Choose an arbitrary cutoff point $0.0<c<1.0$.
      The hypothesis is that alignments with a confidence larger than $c$
        occur more frequently in the lower approximation than alignments
        with a confidence smaller than $c$.
\item Take a set of automatically generated alignment pairs with
        associated confidence measures and take the gold standard or
        reference alignment for the same dataset.
      The hypothesis is that pairs that occur in the lower approximation
        of the alignment appear relatively more often in the gold standard
        than pairs that occur in the higher approximation of the alignment.
\item The accuracy measure $\alpha$ of a reference alignment is generally
        higher than the accuracy measure of an automatically generated
        alignment for the same dataset.
      Or, the accuracy measure is generally higher for identity relations
        that are considered correct by domain experts.
\end{enumerate}

\subsection{Implementation}
\label{sec:implementation}

The implementation built for this paper is deployed as an extension pack
  of the ClioPatria triple store \cite{schreiber_2006}.
For RDF graphs that are loaded in ClioPatria this extension calculates
  the discernibility partition, rough set approximation and accuracy.
The results are visualized using GraphViz and are displayed in a
  Web interface using SVG.
Interactive Ajax code allows the user to click on nodes in the SVG graphic
  to navigate to descriptions of the resource pairs that occur in
  that partition set while not being in the identity relation.
This implementation may facilitate the validation of hypotheses in this
  new experimental setup.

Existing SW libraries implement equivalence relations for common datatypes.
Equivalence is not always in line with identity though.
We have made exceptions for specific values that are known to have
  equivalent non-identities (such as $0$ and $-0$)
  non-equivalent identities (such as $NaN$ for \texttt{float}).\cite{XSD11}
The latter even provides a rare instance of the violation of
  definition \ref{def:identity}.

