\subsection{Path-expressions}
\label{sec:path_expressions}

The indiscernibility predicates in
  definition \ref{def:indiscernibility_predicates}
  were assumed to consist of single RDF predicate terms.
This restriction is rather arbitrary.
For instance,
  it may be the case that resources {\small \texttt{dbp:Amsterdam}}
  and {\small \texttt{openei:Amsterdam}} may not share a single property,
  even though the format is located in {\small \texttt{dbp:Netherlands}}
  and the latter is located in {\small \texttt{openei:Netherlands}}
  (but these are not asserted as being the same country).
However, suppose that {\small \texttt{dbp:Netherlands}}
  borders {\small \texttt{dbp:Germany}}
  and {\small \texttt{openei:Netherlands}}
  borders {\small \texttt{openei:Germany}},
  where {\small \texttt{dbp:Germany}} and
  {\small \texttt{openei:Germany}} are asserted to be identical.
If we generalize the notion of a predicate,
  then {\small \texttt{dbp:Amsterdam}} and
  {\small \texttt{openei:Amsterdam}} will share the property
  of being located in a country that borders Germany.
Obviously, Brussels and Brno share this propery as well,
  so Brussels, Brno and Amsterdam will be indiscernible with respect to
  this predicate (taken in isolation), but the property is at least
  able to discern Brussels, Brno and Amsterdam from
  Stuttgart, Portland, and The Netherlands.

The notion of a predicate, can be easily generalised
  so that is can be denoted by a sequence of RDF predicate terms.
Such sequences are called \emph{path-expressions} in RDF. 

\begin{comment}
For each sequence of predicate terms $\tuplerange{p_1}{p_n}$
  we assume a functional mapping
  $f_{\tuplerange{p_1}{p_n}} : S_G \rightarrow \powerset{O_G}$,
  called the property sequence mapping
  \mbox{(definition \ref{def:generalized_property_map})}.

\begin{definition}[Property sequence map]
\label{def:generalized_property_map}
\begin{align}
  f_{\tuplerange{p_1}{p_n}}(s)
\,=\,
  \bigsetdef{o \in O_G}{
    \exists_{\range{x_0}{x_n}}(
      x_0 = s \land x_n = o \land\nonumber\\
      \bigwedge_{i=0}^{n-1}\nolimits
          \pair{I(x_i)}{I(x_{i+1})}
        \in
          \bigcup_{p \in \equivset{p_{i+1}}}\nolimits \mathit{Ext}(I(p))
    )
  }\nonumber
\end{align}
\end{definition}

By using property sequence maps,
  the definition for generalized indiscernibility predicates
  is only slightly more complex than its simplified version.

\begin{definition}[Indiscernibility criteria]
\label{def:indiscernibility_criteria}
\begin{align}
  \indp_{\approx}(\set{\range{x_1}{x_n}})
=
  \setdef{
    \tuplerange{p_1}{p_n} \in P_G^n
  }{\nonumber\\
      \exists \range{p_1^1}{p_1^n} \in \equivset{p_1},
    \ldots,
      \exists \range{p_m^1}{p_m^n} \in \equivset{p_n}
    (\nonumber\\
        \equivset{f_{\tuplerange{p_1^1}{p_m^1}} (x_1)}
      =
        \ldots
      =
        \equivset{f_{\tuplerange{p_1^n}{p_m^n}} (x_n)}
    )
  }\nonumber
\end{align}
\end{definition}
\end{comment}
