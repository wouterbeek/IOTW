\subsection{Generic properties and property maps}

\subsection{Motivation}

We want to generalize the 

When we look at concrete datasets we see that the resources that make up
  an identity pair do not often share a lot of properties.
The pragmatics reflects this, as identity statements are often used
  to link to other resources and import statements
  (i.e. properties asserted of a resources)
  that are not asserted of the former resource.
However, while there is not too much overlap in properties,
  resources are oftentimes connected at a deeper level,
  via a path that spans two or more properties.

The reason for wanting to extend properties 

\subsection{Conceptual}

For instance the property ``is spoken in'' is denoted by
  the predicate term \verb|IIMBTBOX:spoken_in| in the IIMB dataset.
This relates natural languages to the countries in which they are spoken.

We generalize the notion of a property,
  so that is can be denoted by a sequence of predicate terms.
For example the property ``is spoken in a country whose capital is''
  is denoted by
  $\tuple{\texttt{IIMBTBOX:spoken\_in},\texttt{IIMBTBOX:has\_capital}}$,
  relating natural languages to the capitals of the countries in which
  they are spoken.

\subsection{Theoretical}

For each sequence of predicate terms $\tuplerange{p_1}{p_n}$
  we assume a functional mapping
  $f_{\tuplerange{p_1}{p_n}} : S_G \rightarrow \powerset{O_G}$,
  from subject terms into sets of object terms
  (def. \ref{def:generalized_property_map}).

%$f_{\tuple{\verb|IIMBTBOX:spoken_in|}}$
%$f_{\tuple{\verb|IIMBTBOX:spoken_in|,\verb|IIMBTBOX:has_capital|}}$

\small
\begin{definition}[Generalized property map]
\begin{align}
\label{def:generalized_property_map}
  f_{\tuplerange{p_1}{p_n}}(s)
\,=\,
  \bigsetdef{o \in O_G}{
    \exists_{\range{x_0}{x_n}}(
      x_0 = s \land x_n = o \land\nonumber\\
      \bigwedge_{i=0}^{n-1}\nolimits
          \pair{I(x_i)}{I(x_{i+1})}
        \in
          \bigcup_{p \in \equivset{p_{i+1}}}\nolimits \mathit{Ext}(I(p))
    )
  }\nonumber
\end{align}
\end{definition}
\normalsize

When in the following we talk about properties and property maps,
  we mean generalized properties and generalized property maps
  of arbitrary depth.

The indiscernibility criteria for generic properties
  are slightly more complex.

\small
\begin{definition}[Indiscernibility criteria]
\label{def:indiscernibility_criteria}
\begin{align}
  \indp_{\approx}(\set{\range{x_1}{x_n}})
=
  \setdef{
    \tuplerange{p_1}{p_n} \in P_G^n
  }{\\
      \exists_{\range{p_{11}}{p_{1n}} \in \equivset{p_1}},
    \ldots,
      \exists_{\range{p_{m1}}{p_{mn}} \in \equivset{p_n}}
    (\nonumber\\
        \equivset{f_{\tuplerange{p_{11}}{p_{m1}}} (x_1)}
      =
        \ldots
      =
        \equivset{f_{\tuplerange{p_{1n}}{p_{mn}}} (x_n)}
    )
  }\nonumber
\end{align}
\end{definition}
\normalsize

