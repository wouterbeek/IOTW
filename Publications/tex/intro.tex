\section{Introduction}
\label{sec:introduction}

Identity relations are at the foundation of the Linked Open Data initiative
  and of the Semantic Web in general \cite{bizer_cyganiak_heath_2007}.
They allow the interlinking of alternative descriptions of the same thing.
However, the traditional notion of identity
  (expressed by \verb|owl:sameAs| \cite{motic_paterschneider_grau_2012})
  is often problematic, e.g. when objects are considered the same in some
  contexts but not in others.
The standing practice in such cases is to use weaker relations of relatedness
  (e.g., \verb|skos:related|).
Unfortunately, this limits reasoners in drawing inferences.

According to the traditional semantics of the identity relation,
  identical terms can be replaced for one another in all non-modal contexts
  \emph{salva veritate}.
Practical uses of \verb|owl:sameAs| are known to violate this strict condition
  \cite{halpin_hayes_2010,halpin_hayes_mccusker_mcguinness_thompson_2010}.

Identity is often thought of as having the exact same properies.
This statement is known as the Principle of indiscernibility
(see \ref{principle:indiscernibility_of_identicals})
and has been attributed to Leibniz.\cite{TODO}
% Gottfried Wilhelm Leibniz, section 9, Discourse on Metaphysics.

\small
\begin{principle}[Indiscernibility of identicals]
\label{principle:indiscernibility_of_identicals}
\[
    a = b
  \rightarrow
    \forall_{\phi \in \Phi} \phi(a) = \phi(b)
\]
\end{principle}
\normalsize

Although the principle provides necessary and sufficient conditions
  for identity, it does not point towards an automated procedure
  for enumerating the extension of the identity relation.
Due to its circular nature the set of properties includes
  ``being identical to $x$'' (for every resource $x$).
But even though this priciple does not
  allow a positive identification of identity pairs,
  it does provide an exclusion condition,
  namely resources that are known to not share some property
  are also known to not be identical.

\subsection{Generic problems of identity}

Identity poses several problems which are not specific to the SW.

Firstly, identity does not hold across (all) modal contexts,
  allowing Louis Lane to believe that Superman saved her,
  without her knowling that Clark Kent saved here.

Secondly, identity seems to be a relative concept\cite{Geach},
  allowing two medicines to be the same chemical \emph{substance}
  while not being the same \emph{medicine}.

Thirdly, identity over time poses problems,
  since a ship may be considered the same
  even though all the components from which it is built
  have been changed over the course of time.\cite{}

Lastly, there is the problem of identity under counterfactual assertions
  such as ``If Obama would have been bron outside of the US,
  then he would not have been president of the US today.''\cite{Kripke1980}

\subsection{SW-specific problems of identity: Semantics}

Besides the generic problems of identity,
  there are problems that are specific to the SW.
The first SW-specific problem follows from its semantics;
  the second one from its pragmatics.

In the context of the SW
  \footnote{
    \begin{definition}[Semantics of \emph{owl:sameAs}]
    \label{def:owl_sameAs}
    $\langle a_1, a_2 \rangle \in Ext(I(\texttt{owl:sameAs})) \iff a_1 = a_2$
    \end{definition}
  },
  identity assertions are extra strong
  because of the Open World Assumption.
Stating that two resources are the same
  implies that from now on no new property can be added
  to only one of those resources.
For instance, when one SW contributor claims that
  medicines $a$ and $b$ are the same
  based on them having the exact same chemical composition,
  this prohibits another SW contributor from stating that
  $a$ and $b$ are produced by different companies
  without her introducing an inconsistency.
Formulated in terms of
  principle \ref{principle:indiscernibility_of_identicals},
  on the SW the set $\Phi$ contains
  all properties that can \emph{possibly} be expressed
  in the modeling language.
This set is obviously much bigger than the actual vocabulary
  of any in-use dataset.

Moreover, whether or not two resources share the absence of a \emph{property}
  (i.e., a property of the form ``does not have the property $\phi$''),
  cannot be concluded based on the absence of a \emph{property assertion}.
Such `negative knowledge' must be provided explicitly
  using class restrictions.

\subsection{SW-specific problems of identity: Pragmatics}

Besides the semantic SW-specific problem,
  we can identify a pragmatic SW-specific problem of identity.
The SW is not only a formal model,
  it is also a social construct which evolves over time
  (like the syntactic Web).
Modelers are known not to conform to
  the strict semantics of identity\footnote{
    At the AAAI Fall Symposium 2013 track on Semantics for Big Data
    the problem of identity was considered to be one of the most
    relevant problems facing the Semantic Web today.
  },
  and it is unlikely that all who contribute to the SW are able to
  quantify over the entire realm of possible properties
  (i.e., $\Phi$ in principle \ref{def:principle_of_indiscernibility}).
Practice shows that two modelers sometimes have different opinions,
  the one claiming resources to be identical,
  whereas another considers them to be only (closely) related.

When we look at the SW as a social machine\cite{TODO},
% REF http://sociam.org/www2013/
% WWW2013 Workshop
% The Theory and Practice of Social Machines
  we observe that modelers have different opinions about
  whether two resources are identical or not.
Being a social and symbolic system, the SW has both semantics and pragmatics.

The pragmatics of the SW are encoded in the open-ended collection of
  common practices that effectuate the creation and usage of SW content.
An example of such an encoded common practice are the five `stars'
  of Linked Open Data (LOD) publishing.\cite{TODO}
These five `stars' are effectively five \emph{maxims} that specify
  (part of) the pragmatics of LOD publishing,
  much like the Gricean maxims specify
  (part of) the pragmatics of natural language discourse.\cite{TODO}
  % Grice

The fifth `star' or maxim reads:
``Link your data to other people\'s data to provide context.''
Given that RDF links are often specified using
the \verb|owl:saveAs| predicate\cite{void},
%``RDF links often have the owl:sameAs predicate.'' [VoID]
we conclude that the pragmatics of the SW are orthogonal to its semantics.

The pragmatics of the SW, observed in contemporary common practices,
  states that by asserting identity between resources,
  more context is added for those resources
  (since this results in more triples being asserted about the same resource).
At the same time we see that the strict semantics of identity
  gets violated once the context in which a resource was created
  is extended (by identity assertions) beyond this original context.
This is true regardless of whether `context' is defined in terms of
  `intended use', `domain', `time', or `modality'.

From the social point of view,
  we see that the requirements on SW modelers
  are unreasonably high when they are required to
  assert identity links in accordance with the semantics.
At the same time the pragmatics states that modelers should
  make those links in order to place their knowledge into context.

\subsection{Research goals}
\label{sec:research_goals}

In developing our approach we have the following research goals:
\begin{enumerate}
\item In an identity relation the pairs all look the same.
      We want to characterize subrelations of an identity relation in terms
      of the predicates that occur in the schema of the dataset.
\item Based on an existing identity relation we want to give semantically
      motivated suggestions for extending/limiting the identity relation.
\item We want to assess the quality of an identity relation based on
      the consistency with which it is applied to the data.
\end{enumerate}

