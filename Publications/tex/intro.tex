\section{Introduction}
\label{sec:introduction}

Identity relations are a cornerstone of logic-based knowledge representation.
They allow to state and relate properties of an object
  using multiple names for that object, and conversely,
  they allow to infer that different names actually refer to the same object.

In particular, identity relations are
  at the foundation of the Linked Open Data initiative
  and the Semantic Web (SW) in general \cite{BizerCyganiakHeath2007}.
The SW consists of many different sets of assertions
  that are published on the Web by different authors in different locations,
  often using different names for the same object.
Identity relations then allow the interlinking of these multiple descriptions
  of the same thing.
For example, statements about Amsterdam in the DBpedia dataset
  (where Amsterdam is referred to as
  \URL{http://dbpedia.org/resource/amsterdam}, abbreviated as
  \URL{dbp:amsterdam})
  can be combined with statements about Amsterdam in GeoNames
  (where Amsterdam is referred to as
  \URL{http://sws.geonames.rg/2759794}), by asserting identity between
  these two names.

However, the traditional notion of identity
  (expressed by \texttt{owl:sameAs} \cite{MotikPaterschneiderGrau2012})
  is often problematic, e.g. when objects are considered the same in some
  contexts but not in others.
The standing practice in such cases is to use weaker relations of relatedness
  (e.g., \texttt{skos:related} \cite{MilesBechhofer2009}).
Unfortunately, these relations suffer from the opposite problem of having
  almost no formal semantics, thereby limiting reasoners
  in drawing inferences.

According to the traditional semantics of the identity relation,
  identical terms can be replaced for one another in all non-modal contexts
  \emph{salva veritate}.
Practical uses of \texttt{owl:sameAs} are known to violate this
  strict condition
  \cite{HalpinHayes2010,HalpinHayesMccuskerMcguinnessThompson2010}.

\begin{comment}
The SW is not only a formal model,
  but is also a social component that evolves over time,
  i.e. it is a social machine cite{Www2013}.
Being a social and symbolic system at the same time,
  meaning on the SW is denoted by its semantics as well as its pragmatics.
\end{comment}

\subsection{Outline}

In the following section,
  we will analyze in some more detail the problems that
  are caused by the traditional notion of identity, both in general,
  and in the SW setting in particular.
After surveying existing work on these problems
  in section \ref{sec:related_work},
  we will then present our approach to the problem of identity
  in section \ref{sec:approach}.
First at a very high level and then in more formal detail.
We illustrate the results of applying our formalism to
  a realistic heterogeneous SW dataset.
In sections \ref{sec:implementation} and \ref{sec:experimental_design}
  we put our formalism to the test
  in an experimental setting,
  showing that it behaves as required.
