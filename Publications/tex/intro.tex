\section{Introduction}
\label{sec:introduction}

Identity relations are at the foundation of the Linked Open Data initiative
  and of the Semantic Web in general \cite{BizerCyganiakHeath2007}.
They allow the interlinking of alternative descriptions of the same thing.
However, the traditional notion of identity
  (expressed by
  {\small \texttt{owl:sameAs}} \cite{MotikPaterschneiderGrau2012})
  is often problematic, e.g. when objects are considered the same in some
  contexts but not in others.
The standing practice in such cases is to use weaker relations of relatedness
  (e.g., {\small \texttt{skos:related}} \cite{MilesBechhofer2009}).
Unfortunately, this limits reasoners in drawing inferences.

According to the traditional semantics of the identity relation,
  identical terms can be replaced for one another in all non-modal contexts
  \emph{salva veritate}.
Practical uses of {\small \texttt{owl:sameAs}} are known to violate this
  strict condition
  \cite{HalpinHayes2010,HalpinHayesMccuskerMcguinnessThompson2010}.

The SW is not only a formal model,
  but is also a social component that evolves over time,
  i.e. it is a social machine \cite{Www2013}.
Being a social and symbolic system at the same time,
  meaning on the SW is denoted by its semantics as well as its pragmatics.

\subsection{Research goals}
\label{sec:research_goals}

In developing our approach we have the following research goals:
\begin{enumerate}
\item In an identity relation the pairs all look the same.
      We want to characterize subrelations of an identity relation in terms
      of the predicates that occur in the schema of the dataset.
\item Based on an existing identity relation we want to give semantically
      motivated suggestions for extending/limiting the identity relation.
\item We want to assess the quality of an identity relation based on
      the consistency with which it is applied to the data.
\end{enumerate}

