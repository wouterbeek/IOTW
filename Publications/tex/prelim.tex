\subsection{Preliminaries}
\label{sec:preliminaries}

$G$ denotes an RDF graph. It consists of a set of ground binary
predicates $p(s,o)$, called ``triples'' in Semantic Web jargon, and often
written as $\triple{s}{p}{o}$. These triples form a graph with all
subjects $s$ and objects $o$ as nodes, and 
each assertion $p(s,o)$ corresponding to a directed edge labelled $p$
between $s$ and $o$. 

We assume that graphs are always closed under
  RDFS and OWL-DL entailment
  (see section \ref{sec:implementation} for details).

We identify subsets of RDF terms based on
  their positional occurrence in triples in $G$:
  $S_G$, $P_G$, and $O_G$ denote the subject, predicate and object terms
  in $G$ respectively.

The interpretation $I$ maps RDF terms onto resources,
  and triples onto truth values.
The extension function $Ext$ maps resources onto pairs of resources.
$I(\triple{s}{p}{o})$ is true iff
  $\pair{I(s)}{I(o)} \in Ext(I(p))$ \cite{Hayes2004}.

A note on terminology: Our use of the word `property' does not coincide
  with the notion of an RDF property, but is closer to the notion
  of a FOL property. Our use of the word `relation' will be close to
  both the notion of a FOL relation and the notion of an RDF property.
  We briefly explain the distinctions.

An RDF property is a resource (i.e., a member of the domain)
  that is the interpretation of
  an RDF term that occurs in the predicate position of a triple.\footnote{
    Notice that we do not use the posibility modality in this formulation.
    Every URI \emph{could} occur in the predicate position of a triple,
      but some URIs are known to denote non-property resources.
  }
As the example above shows, the extension of an RDF property is
  a binary \emph{FOL relation},
  i.e. a subset of the cartesian product of the domain.\footnote{
    In RDF the domain is the set of RDF resources.}

A \emph{FOL property} is a subset of the domain.
The correlate of a FOL property in RDF is a pair that consists of
  a predicate term and an object term (in that order).
The extension of the interpretation of such a pair $\pair{p}{o}$
  is $Ext(I(p))(I(o))$, which is -- indeed -- a set of RDF resources.

In section \ref{sec:path_expressions} we will generalize
  both the concept of a property and the concept of a relation.
Generalized properties will still be similar to FOL properties
  and generalized binary relations will still similar to binary FOL relations,
  but the latter will no longer correspond to a (single) RDF property.

\begin{comment}
$\equivset{x}$ is the equivalence class for $x$
  under equivalence relation $\approx$.
\end{comment}

