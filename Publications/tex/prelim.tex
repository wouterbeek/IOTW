\subsection{Preliminaries}
\label{sec:preliminaries}

RDF terms that occur in the predicate position of an RDF triple are
  \textbf{RDF predicate terms}.
An example from the IIMB dataset is \verb|IIMBTBOX:spoken_in|.
This is not the same as the terms that \emph{can} occur
  in the predicate position of a triple,
  since any URI can occur in this position,
  but some URIs denote non-propery resources.

The interpretation of an RDF predicate term is an \textbf{RDF property},
  which is an RDF resource.

The extension of an RDF property is a \textbf{binary FOL relation},
  i.e. a subset of the cartesian product of the domain
  \footnote{In RDF the domain is the set of RDF resources.}.

A \textbf{FOL property} is a subset of the domain.

The correlate of a FOL property in RDF is a pair
  consisting of a predicate and an object term (in that order).
We call this syntactic construct a \textbf{$PO$-sequence}.
The extension of the interpretation of a $PO$-seq is
  $Ext(I(p))(I(o))$, which is a set of RDF resources.

\begin{itemize}
\item RDF `graph' $G$.
\item RDF subject terms $S_G$.
\item RDF property terms $P_G$.
\item RDF object terms $O_G$.
\item The set of blank nodes $\mathcal{B}$ (definition \ref{def:blank_node}).
\item The set of plain literals $\mathcal{LP}$ (definition \ref{def:plain_literal}).
\item The set of typed literals $\mathcal{LT}$ (definition \ref{def:typed_literal}).
\item The canonical mapping for a datatype $d$, $c_d : LEX_d \rightarrow V_d$.
\item An interpretation function $\mathcal{I}$.
\item Canonical projecting map
  \[
    \pi : (\mathcal{B} \union \mathcal{I})
  \rightarrow
    (\mathcal{B} \union \mathcal{I})/\sim
  \].
\item $\equivset{p}$ is the equivalence class for $p$ under $\approx$.
\end{itemize}

In the following we consider an arbitrary,
  materialized RDF graph $G$ and an identity relation $\approx$ over
  the resources that occur in $G$.
The subject terms of $G$ are denoted by $S_G$, the predicate terms by $P_G$
  and the object terms by $O_G$.

