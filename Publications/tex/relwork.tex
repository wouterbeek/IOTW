\section{Related work}
\label{sec:related_work}

Existing research suggests six different solutions for
  the problem of identity on the SW.

\textbf{[1] Introduce weaker versions of {\small \texttt{owl:sameAs}}}
  \cite{HalpinHayes2010,MccuskerMcguinness2010}.
Candidates for replacement are
  the SKOS concepts
  {\small \texttt{skos:related}} and {\small \texttt{skos:exactMatch}}
  \cite{MilesBechhofer2009}.
The former is not transitive,
  thereby limiting the possibilities for reasoning.
The latter is transitive,
  but can only be used in certain contexts.
It is not defined in what contexts it can be used
  \cite{MilesBechhofer2009}.\footnote{
    For instance, the property {\small \texttt{skos:exactMatch}}
    ``is used to link two concepts, indicating a high degree of confidence
    that the concepts can be used interchangeably across a wide range of
    information retrieval applications.''
  }
\begin{comment}
% SIMILARITY
The problem with using weaker notions such as relatedness,
  is that everything is related to everything in \emph{some} way.}
% Shall we discuss similarity here as well?
% Does similarity differ from relatedness?
\end{comment}

\textbf{[2] Restrict the applicability of identity relations}
  to specific contexts.
In terms of Semantic Web technology, identities are expected to hold
  within a named graph or within a namespace,
  but not necessarily outside of it \cite{HalpinHayes2010}.
\cite{Melo2013} has successfully used the Unique Names Assumption
  within namespaces in order to identify many (arguably) spurious
  identity statements.

\textbf{[3] Introduce additional vocabulary} that does not weaken but extends
  the existing identity relation.
\cite{HalpinHayes2010} mention an explicit distinction that could be made
  between mentioning a term and using a term,
  thereby distinguishing an object and a Web document describing that object.
Other possible extensions of {\small \texttt{owl:sameAs}} might take
  the Fuzzyness and/or uncertainty of identity statements into account.

\textbf{[4] Use domain-specific identity relations}
  \cite{MccuskerMcguinness2010}.
For instance
    ``$x$ and $y$ have the same medical use''
  replaces
    identity in the domain of medicine,
and
    ``$x$ and $y$ are the same molecule''
  replaces
    identity in the domain of chemistry.
The downside to this solution is that domain-specific links are
  only locally valid, thereby limiting knowledge reuse.

\textbf{[5] Change the modeling practice}, possibly in a (semi-)automated way
  by adapting visualization and modeling toolkits to produce notifications
  upon reading SW data, or by posing additional restrictions on the creation
  and alteration of data. For example, adding an RDF link could require
  reciprocal confirmation from the maintainers of the respective datasets.
  \cite{HalpinHayes2010,DingShinavierFininMcguinness2010}
The problem with introducing checks on editing operations,
  is that it violates one of the fundamental underpinnings of the SW;
  namely that on the Web of Data anybody is allowed to say
  anything about anything \cite{AntoniouGrothHarmelenHoekstra2012}.

\textbf{[6] Extract network properties of {\small \texttt{owl:sameAs}}
  datasets} \cite{DingShinavierShangguanMcguinness2010}.
Although this work shows that network analysis can provide insights
  into the ways in which identity is used in the SW,
  these endeavors have not yet been related to the semantics of the
  identity relation.
We believe that utilizing network theoretic aspects in order to
  determine the meaning of identity statements
  would be interesting future research.

What the existing approaches have in common is
  that quite some work has to be done
  (adapting or creating standards, instructing modelers, converting existing
  datasets) in order to resolve only some of the problems of identity.
Our approach provides a way of dealing with the heterogeneous real-world
  usage of identity in the SW that is fully automated and requires
  no changes to standards, modeling practices, or existing datasets.
