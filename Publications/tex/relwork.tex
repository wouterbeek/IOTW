\subsection{Related work}
\label{sec:related_work}

Existing research proposes six different solutions for
the problem of identity on the Semantic Web.
We now briefly discuss each of these.
(1) Introduce weaker versions of \verb|owl:sameAs|
  \cite{halpin_hayes_2010,mccusker_mcguinness_2010}
  (e.g., \verb|skos:related|).
(2) Restrict the applicability of identity relations to specific contexts.
  Identities are expected to hold within a named graph or within a namespace,
  but not necessarily outside of it \cite{halpin_hayes_2010,melo_2013}.
(3) Introduce additional vocabulary that does not weaken but extend
  the existing identity relation \cite{halpin_hayes_2010}.
  For example, allow an explicit distinction to be made between mentioning
  a term and using a term
  (e.g., a car and a Web document describing that car).
(4) Add domain-specific weaker versions of the identity relation
  \cite{mccusker_mcguinness_2010} (e.g., ``have the same medical use''
  is weaker than ``are the same molecule'').
(5) Adapt the modeling practice, possibly in a (semi-)automated way
  by adapting visualization and modeling toolkits to produce notifications
  upon reading SW data, or by posing additional restrictions on the creation
  and alteration of data. For example adding an RDF link could require
  reciprocal confirmation from the maintainers of the respective datasets.
  \cite{halpin_hayes_2010,ding_shinavier_finin_mcguinness_2010}

Other related research focusses on the extraction of network properties of
  \verb|owl:sameAs| datasets \cite{ding_shinavier_shangguan_mcguinness_2010},
  but these endeavors are not yet related to the semantics of the
  identity relation.

What these approaches have is common is that quite some work has to be done
  (adapting or creating standards, instructing modelers, converting existing
  dataset) in order to resolve some of the problems of identity.
Our approach provides a way of dealing with the heterogeneous real-world usage
  of identity in the Semantic Web that is fully automated and that requires
  no changes to standards, modeling practices, or existing datasets.

In modeling the problem arises that identical resources allow reasoning

Since all resources are related

An example from the SW is the property \verb|skos:exactMatch|,
which is used to indicate that two conceptual resources in
different concept schemes are sufficiently similar that
they can be used interchangeably
in an information retrieval application.
